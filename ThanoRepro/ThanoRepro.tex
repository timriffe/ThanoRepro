
\documentclass{article}
\usepackage{amsmath}
\usepackage{caption}
\usepackage{placeins}
\usepackage{graphicx}
\usepackage{subcaption}
\usepackage{natbib}
\bibpunct{(}{)}{,}{a}{}{;} 
\usepackage{url}
\usepackage{nth}
% for the d in integrals
\newcommand{\dd}{\; \mathrm{d}}

\usepackage[top=2in, bottom=1.5in, left=1in, right=1in]{geometry}

% matrix numbering
\usepackage{trivfloat}
\trivfloat{matrix}
\floatstyle{plaintop}
    \restylefloat{matrix}
\AtBeginDocument{\numberwithin{matrix}{section}}
% end preamble
%-------------------------------------------------------

\begin{document}

\title{Renewal and stability in populations structured by remaining
years of life}
\author{Tim Riffe \\ University of California, Berkeley}
\maketitle

\begin{abstract}
We transform data classified by chronological age into data classified by
remaining years of life (thanatological age). A model for population renewal and
the corresponding projection matrix are presented for populations structured by
thanatological age. Period results are derived using all available data from the
HMD and HFD. We compare the intrinsic growth rate, $r$, as derived from the
classic Lotka equation versus that derived on the basis of thanatological age.
We also compare some transient indicators between the two
models. Empirical results suggest that $r$ from the thanatological model tends to be less erratic than
Lotka's $r$, and the trajectory to stability tends to be faster with less
oscillation.
\end{abstract}

*This is a work in progress. All findings are preliminary at this time, so
please don't cite without permission of the author.
\vspace{2em}

All demographic forces vary over age in known and regular ways. Information on
such forces, along with the size and age structure of a population allows the
demographer to make predictions about the future size and structure of a population. In this
paper, we explore some formal demographic consequences of a particular
redefinition of age. Instead of counting age as the time passed between birth and the present (or
some other moment), consider age as the amount of time left from the present
until death. Individuals in this case move in the same direction along
imaginary life lines, but the reference point for age is now at the end of the
life line instead of at the beginning. Of course, such information cannot be
requested of survey respondents or register data, because one's time at death is
typically unknown. Instead we approximate this perspective using a similar
strategy to that used by \citet{miller2001increasing} or
\citet{lee2002approach}, which uses lifetables and aggregated
population counts.

Temporal perspectives and rescalings are abundant in demography already,
but questions are left unanswered. Likely some demographic phenomena are best
described as a function of time since birth, and others of time until death,
while still others can be a function of both to some degree. This author
supposes that fertility and reproduction can be a function of both age
perspectives, and that the question of how to partition forces and events over
time is a difficult one. Thanatological reproduction sits at the opposite
extreme of chronological reproduction, the Lotka-Euler renewal
model. By specifying a model of thanatological renewal, we both expand the
toolbox available to approach complex temporal structure and we provide a set of broad
findings to help understand the implications of temporal structure. 

\section*{From chronological to thanatological age}

Our point of departure is a little formula that gives the probability of
dying at some age $n$ years in the future, given survival to some
earlier age, $a$. For age 0 this is simply the $d(a)$ column of the lifetable,
standardized to sum to 1. For higher ages, we simply condition on survival:

\begin{equation}
\label{eq:vaupel1}
f(n | a) = \mu(a+n)\frac{l(a+n)}{l(a)}
\end{equation}

A convenient discretization of \eqref{eq:vaupel1} is to simply work with the
$d_x$ column of the lifetable

\begin{equation}
d_{x+n, x} = \frac{d_{x+n}}{\sum_{i=x}^\omega d_i }
\end{equation}
where $\omega$ is the highest age.\footnote{In later formulas we just use $l_x$
in the denominator to reduce clutter, but the above sum over $d_x$ will eliminate
rounding error in case one is working from a published life table.} One
produces in this way a probability density function over future death times for each age, and the $d_{x+n, x}$- weighted average of $n$ for a given age $x$ will simply produce the familiar remaining life expectancy column of the lifetable, $e_x$. Since $1 = \sum_{n=0}^{\omega - x}d_{x+n, x}$ for each $x$, we can use this to
decompose age-classified demographic data. For example, we can break down a
population pyramid into remaining years categories, like in
Figure~\ref{fig:USdecomp}, which shows this exercise for 2010 US data. The fun starts when we proceed to regroup
the data by $n$ instead of collapsing back to $x$. Now instead of remaining lifetime 
heterogeneity within each age group we have age heterogeneity within each
discrete remaining years of life. Figure~\ref{fig:USrecomp} shows this
\textit{re}composition for the same 2010 US data.

\begin{figure}
	\caption{2010 US population}
	\begin{center}
	\begin{subfigure}{.45\textwidth}
		\caption{Chronological age-structure decomposed by thanatological age.}
		\label{fig:USdecomp}
		\includegraphics[scale=.6]{Figures/US2010Age.pdf}
	\end{subfigure}
	~
	\begin{subfigure}{.45\textwidth}
		\caption{Thanatological age-structure decomposed by chronological age.}
		\label{fig:USrecomp}
		\includegraphics[scale=.6]{Figures/US2010Thano.pdf}
	\end{subfigure}
	\\
	\small{*Population and mortality data from the HMD.}
	\end{center}
\end{figure}

If $P_x$ is an age-classified discrete population count, we can derive $P_y$, a
remaining-years-classified population count, as follows:

\begin{align}
\label{eq:transform}
P_y =& \sum_{x=0}^\omega P_x \frac{d_{x+y}+d_{x+y+1}}{l_x+l_{x+1}}
\intertext{and of course}
P =& \sum P_y = \sum P_a
\end{align}

We refer to \eqref{eq:transform} as the thanatological age
transformation\footnote{I have on good authority that Ken Wachter
coined the term \textit{thanatological} age.}, and this complements the notion
of chronological age. It bears repeating that thanatological age is merely probabilistic, since it
refers to the future according to particular assumptions about the survival
curve. As such, Figure~\ref{fig:USrecomp} is projective in nature, a
forward-looking glance at a population given the age-structure and lifetable of a particular moment, whereas
Figure~\ref{fig:USdecomp} is reflective in nature, since a population's
age-structure is mostly the fruit of past fertility, but also migration and to a
lesser degree attrition. In a different paper we explore population structure by remaining years of life
for its own sake. Note that any age-classified count can be
reclassified in this way, as long as you have a lifetable that plausibly
represents the population whose data you wish to reclassify. This means we can
derive thanatological fertility rates, $F_y$, by applying \eqref{eq:transform}
to age-classified birth counts, $B_x$, to get $B_y$ and again to
exposure-to-risk, $E_x$ (here we take exposures from all ages), to get $E_y$ and then dividing:

\begin{equation}
F_y = \frac{B_y}{E_y}
\end{equation}

We are interested to use thanatological fertility rates, $F_y$ in
models of population renewal, rather than to explicitly study the nature of these rates.
Nonetheless it is best not to throw such rates into a model blindly, so we
offer a schematic overview of some of their characteristics.
Figure~\ref{fig:FySpaghetti} represents the full variety in
female thanatological period fertility rates that can be found for all years
of data that overlap in the HFD and HMD.\footnote{There are as of this writing
1834 population-years of overlap between the HMD and HFD, including a wide
variety of fertility and mortality combinations.} Figure~\ref{fig:FxSpaghetti} gives ASFR for the same
populations and years as a more familiar reference, but note that both scales are different! With no indication of particular populations or time series, one already concludes that thanatological fertility rates have a characteristic shape and are not random, erratic or informationless: there is a pattern to fertility rates by remaining years of life, and it varies over time and between populations within some range of normality. Thanatological fertility rates have a wider distribution than do chronological rates. Note that this spaghetti plot includes both fertility booms and busts, as
well as some mortality crises (1918, WWII). Some patterns to note, but which we
do not separate graphically here, are that the left tail (a gauge of how
orphan-prone a population is) has tended to fall over time. All populations have
shown a rightward shift in both the mean and modal female thanatological age at
birth (TAB) over time (that's unambiguously a good thing). Several populations
now show modal TABs of over 60 years. Thanatological total fertility rates track
standard PTFR rather closely, but tend to be somewhat higher. 

\begin{figure}[h!]
	\caption{Chronological and thantological fertility rates, all 1600
	country-year combinations present in both the HFD and HMD.* Note different x
	and y scales.}
	\label{fig:Fxcompare}
	\begin{center}
	\makebox[\textwidth]{
	\begin{subfigure}{.45\textwidth}
		\caption{Chronological period fertility rates.}
		\label{fig:FxSpaghetti}
		\includegraphics[scale=.45]{Figures/FxSpaghettiDraft.png}
	\end{subfigure}
	~
	\begin{subfigure}{.45\textwidth}
		\caption{Thanatological period fertility rates.}
		\label{fig:FySpaghetti}
		\includegraphics[scale=.45]{Figures/FySpaghettiDraft.png}
	\end{subfigure}
	}
	\\
	\end{center}
	\begin{tiny}
	*AUT, $1951-2010$; BGR, $1947-2009$; BLR, $1964-2009$; CAN, $1921-2007$; 
	CHE, $1932-2011$; CZE, $1950-2011$; DEUTE, $1956-2010$; DEUTNP, $1990-2010$; 
	DEUTW, $1956-2010$; ESP, $1922-2006$; EST, $1959-2010$; FIN, $1939-2009$; 
	FRATNP, $1946-2010$; GBR\_NIR, $1974-2009$; GBR\_NP, $1974-2009$; GBR\_SCO,
	$1945-2009$; GBRTENW, $1938-2009$; HUN, $1950-2009$; IRL, $1955-2009$; JPN, $1947-2009$; 
	LTU, $1959-2010$; NLD, $1950-2009$; NOR, $1967-2009$; PRT, $1940-2009$; 
	RUS, $1959-2010$; SVK, $1950-2009$; SVN, $1983-2009$; SWE, $1891-2010$; 
	TWN, $1976-2010$; UKR, $1959-2006$; USA, $1933-2010$
	\end{tiny}
\end{figure}

In general, such fertility rates will not be palatable for purposes of
projection, unless the demographer believes that their empirical regularity is
somehow stronger than typical ASFR --- an argument we do not make. There may be
alternative ways of defining thanatological fertility rates, for instance
\textit{after} stability, which would make this model component more conformable
with the Lotka renewal model. First let us define the basic model.

\section*{The thanatological renewal model}
For the population model that follows, we are most interested in using the kind
of fertility rates shown in Figure~\ref{fig:FySpaghetti}, and we refer to a
unisex population. Assuming constant vital rates, the births for the present
year are given by:
\begin{align}
B(t) =& \int _0^\infty P(y,t)F(y) \dd y = \int _0^\infty P(a,t)F(a) \dd a
\intertext{where $P(a), P(y)$ are population counts and $F(y), F(a)$ are exact
specific fertility probabilities (rates), and $d(a)$ is the continuous
lifetable death distribution with radix of 1, so $d(a)/l(0)$. The
thanatological integral can be broken down back in terms of chronological age.} =&
\int_{y=0}^\infty
\int_{a=0}^\infty F(y)\frac{P(a,t)d_(a+y)}{l(a)} \dd a \dd y 
\intertext{We can relate the present population to past
births with $P(a,t) = B(t-a)l(a)$.} 
\label{eq:ergostep1}
=& \int_{y=0}^\infty \int_{a=0}^\infty F(y) B(t-a)d(a+y)\dd a \dd y
\intertext{And eventually-- a bit quicker than is the case for chronological
age-- strong ergodicity will assert itself, and $B(t)$ will be related to
$B(t-a)$ according to a constant factor $e^{ra}$, where $r$ is the familiar
intrinsic rate of growth.}
\label{eq:ergostep2}
=& \int_{y=0}^\infty \int_{a=0}^\infty F(y) B(t)e^{-ra}d(a+y)\dd a \dd y
\intertext{Divide out $B(t)$ to get back a familiar-looking renewal equation.}
\label{eq:thanoren}
1 =& \int_{y=0}^\infty \int_{a=0}^\infty F(y) d(a+y)e^{-ra}\dd a \dd y
\end{align}
Now compare this to Lotka's chronological formulation
\begin{equation}
\label{eq:lotka}
1 = \int_{a=0}^\infty F(a)l(a)e^{-ra}\dd a
\end{equation}
and note that these are really quite similar, since $\int _{y=0}^\infty
d(a+y)\dd a = l(a)$. For intuition, notice that $l(a)$ is here split
up into pieces of $d(a)$, and imagine a 2D surface of these, where one axis is chronological age and the
other axis is thanatological age. For the chronological case \eqref{eq:lotka},
we multiply chronological age-specific fertility rates over one margin, and for
the thanatological case over the other margin. There is a strong parallel here
with the case of two-sex age-specific fertility rates. As in the case of
divergence between male and female single-sex models, there will always be divergence between
single-sex models under chronological versus thanatological age. Even though the
modeled population stocks are in a way commensurable, the rates used, $F(a)$
versus $F(y)$ are calculated on the basis of differently distributed
denominators $E(a)$ versus $E(y)$.

There are gaps in the above line of development, since the jump from
\eqref{eq:ergostep1} to \eqref{eq:ergostep2} (strong ergodicity) is unproven,
although it is rather intuitive, given the much greater density of connections
within the model; persons from nearly any thanatologcal age can produce
offspring that can have any other thanatological age. In this sense, the
smoothing mechanism must be much stronger than that for chronoloical age, at
least in most cases. A further conjecture is that the jump will also hold for
the case of weak ergodicity. The mechanisms at play unfold in the same way as those so intuitively described by \citet{arthur1982ergodic}, and said proof may apply here without further modification. A proof of the uniqueness of the solution to \eqref{eq:thanoren} is given in Appendix~\ref{app:A}.

Once one finds $r$ from \eqref{eq:thanoren}, other familiar stable population
parameters can be calculated. For instance, we may calculate the mean
thanatological generation time, $T^y$, as:
\begin{equation}
\label{eq:Ty}
 T^y =  \frac{\int _{y=0}^\infty \int _{a=y}^\infty y e^{-ra} d(a) F(y) \dd a
\dd y}{\int _{y=0}^\infty \int _{a=y}^\infty e^{-ra} d(a) F(y) \dd a \dd y}
\end{equation}
The net reproduction rate, $R_0$ is related by, e.g.,
\begin{equation}
\label{eq:R0fromTy}
R_0 = e^{r T^y}
\end{equation}
The birth rate, $b$, is given by
\begin{equation}
\label{eq:eybrate}
b = \frac{1}{\int _{y=0}^\infty \int _{a=y}^\infty e^{-ra} d(a) \dd a
\dd y}
\end{equation}
The stable age structure, $c$, where $c_y$ is the
proportion of the stable population with remaining years to live $y$, is given
by
\begin{equation}
\label{eq:cy}
c(y) = b \int _{a=y}^\infty e^{-ra} d(a) \dd a
\end{equation}
Other stable population quantities may be estimated by similarly translating the
various common definitions (e.g., in the glossary of \citet{coale1972growth}) to
the present perspective. We will focus on the main model rather than on these.

The thanatological renewal model is coherent, but the idea may seem strange.
What should one imagine under the model of thanatological population renewal? 
A useful mnemonic bases itself on Figures~\ref{fig:USdecomp} and
\ref{fig:USrecomp}. In the age-structured model, new generations
appear at the bottom of the pyramid, and move up one rung per year. All ages 
are subject to decrement. In the thanatological \textit{leaf}, each birth cohort increments to
the population over the whole range of thanatological age according to $d(a)$,
as seen in \eqref{eq:thanoren}. Birth cohorts become the layers seen in
Figure~\ref{fig:USrecomp}. Each horizontal step is a death cohort, and these
move one step down the pyramid each year without any decrement (indeed
incrementing) until reaching the very bottom. In short, the locations of
increment and decrement, and the direction of movement are all switched.

\section*{The thanatological projection matrix}
These descriptions can be made more explicit by hashing out the projection
matrix that corresponds to the thanatological renewal model. As with the
age-structured Leslie matrix, the thanatological projection matrix,
$\textbf{Y}$, is square and of dimension $n \times n$, where $n$ is the number
of remaining-years classifications into which the population is divided. 
The matrix contains elements for survival and elements for fertility. Unlike
Leslie matrices, $\textbf{Y}$ is not sparse, but is populated primarily with non-zero entries.

Of interest is that mortality occurs in only the population class with zero
remaining years of life. Thanatological age 1 in year $t$ moves to 0 in year
$t + 1$. Thus, instead of in the subdiagonal, we place survival in the
superdiagonal. All survival values are 1, since there is no decrement, and the
upper-left corner of this superdiagonal contains a 0, for full decrement.
We illustrate using a 6$\times$6 matrix. The survival component of $\textbf{Y}$ is organized as in
Matrix~\ref{matrix:Ysurv}.

\begin{matrix}[h!]
\centering
\caption{Survival component of unisex thanatological projection matrix,
$\textbf{Y}$}
\label{matrix:Ysurv}
$\bordermatrix{{e_y } & 0_t & 1_t & 2_t & 3_t & 4_t & 5_t\cr 
                0_{t+1} & 0    &  1   & 0    & 0    & 0    & 0   \cr
                1_{t+1} & 0    &  0   & 1    & 0    & 0    & 0   \cr 
                2_{t+1} & 0    &  0   & 0    & 1    & 0    & 0   \cr 
                3_{t+1} & 0    &  0   & 0    & 0    & 1    & 0   \cr 
                4_{t+1} & 0    &  0   & 0    & 0    & 0    & 1   \cr
                5_{t+1} & 0    &  0   & 0    & 0    & 0    & 0   }$
\end{matrix}

 Fertility inputs to the matrix are derived from single-sex thanatological
 fertility and the lifetable $d_a$ distribution, where $a$ indexes age, but is
 equal to $y$, remaining years of life for persons aged 0. Fertility
 in a thanatologically structured population occurs in all but the highest
 remaining years classes. For our example, say that fertility is observed in
 classes 0-4, while the final class has no fertility, where $F_y$ indicates the
 fertility probability for class $y$ in the year $t$ entering population (in the 
 matrix columns). Each $F_y$ is then distributed according to
 $d_a$ with no further translation, since the $d_a$ lifetable column already
 refers to age 0. Thus, the fertility entry in row $m$ and column $n$ of
 $\textbf{Y}$ will be $F_n \cdot d_m$. We assume that those dying over the course of year
 $t$ (the first column) are exposed to fertility for half of the
 year,\footnote{One might be tempted to not allow for fertility at all for
 females dying in year $t$, but recall that fertility is measured in the moment of
 birth, and not conception.}
 and so discount the fertility entry accordingly. Further, infant mortality, 
 $F_y \cdot d_0$, located in the first row, must also be discounted, since part
 of the mortality will occur in the same year $t$ and the rest in year $t + 1$. 
 The first row of fertility must be further discounted by a factor, $\lambda$,
 in order to account for the fact that infant mortality is higher in the lower Lexis 
 triangle than in the upper: of those infants who die in the first year of life, a proportion equal to
 $\lambda$ do not make it to December \nth{31} of the calendar year in which
 they were born.\footnote{$\lambda$ can be derived directly from death counts
 data classified by Lexis triangles. In the US, $\lambda$ has behaved similarly
 for males and females, falling steadily from around $0.9$ in 1969 to $0.86$
 around 1990, since which time it has steadily risen to around $0.87$. That is
 to say, $\lambda$ has varied, but not drastically. These numbers are just meant to give a
feel for the ranges that $\lambda$ can be expected to receive. If the demographer
 does not have information to derive $\lambda$ directly, ad hoc or semidirect
 methods may be used to assign a reasonable proportion. } The
 fertility component of $\textbf{Y}$ is then composed as in Matrix~\ref{matrix:Yfert}.

\begin{matrix}[h!]
\centering
\caption{Fertility component of unisex thanatological projection matrix,
$\textbf{Y}$}
\label{matrix:Yfert}
$\bordermatrix{
  {e_y } \vspace{.6em}&                0_t  & 1_t  & 2_t  & 3_t  & 4_t  & 5_t\cr 
   0_{t+1} \vspace{.6em}& (1-\lambda) \tfrac{F_0d_0}{2} & (1-\lambda) F_1d_0 & (1-\lambda)
   F_2d_0 & (1-\lambda) F_3d_0 & (1-\lambda) F_4d_0 & 0 \cr 
   1_{t+1} \vspace{.6em}& \tfrac{F_0d_1}{2} & F_1d_1 & F_2d_1 & F_3d_1 & F_4d_1
   & 0   \cr 2_{t+1} \vspace{.6em}& \tfrac{F_0d_2}{2} & F_1d_2 & F_2d_2 & F_3d_2 & F_4d_2
   & 0   \cr 3_{t+1} \vspace{.6em}& \tfrac{F_0d_3}{2} & F_1d_3 & F_2d_3 & F_3d_3 & F_4d_3
   & 0   \cr 4_{t+1} \vspace{.6em}& \tfrac{F_0d_4}{2} & F_1d_4 & F_2d_4 & F_3d_4 & F_4d_4
   & 0   \cr 5_{t+1} \vspace{.6em}& \tfrac{F_0d_5}{2} & F_1d_5 & F_2d_5 & F_3d_5 & F_4d_5
   & 0   }$
\end{matrix}

The survival and fertility components of $\textbf{Y}$ add together elementwise,
thus the full 6$\times$6 matrix is composed as in Matrix~\ref{matrix:Y}.

\begin{matrix}[h!]
\centering
\caption{A full unisex thanatological projection matrix, $\textbf{Y}$} 
\label{matrix:Y}
$\textbf{Y} = \bordermatrix{
  {e_y } \vspace{.6em} & 0_t  & 1_t  & 2_t  & 3_t  & 4_t  & 5_t\cr 
  0_{t+1} \vspace{.6em}&  (1-\lambda) \tfrac{F_0d_0}{2} & (1-\lambda) F_1d_0 + 1 &
  (1-\lambda) F_2d_0 & (1-\lambda) F_3d_0 & (1-\lambda) F_4d_0 & 0 \cr 
    1_{t+1} \vspace{.6em}& \tfrac{F_0d_1}{2} & F_1d_1 & F_2d_1 + 1 & F_3d_1 & F_4d_1 & 0 \cr 
    2_{t+1} \vspace{.6em}& \tfrac{F_0d_2}{2} & F_1d_2 & F_2d_2 & F_3d_2 + 1 & F_4d_2 & 0 \cr 
   3_{t+1} \vspace{.6em}& \tfrac{F_0d_3}{2} & F_1d_3 & F_2d_3 & F_3d_3 & F_4d_3 + 1 & 0 \cr 
   4_{t+1} \vspace{.6em}& \tfrac{F_0d_4}{2} & F_1d_4 & F_2d_4 & F_3d_4 & F_4d_4 & 1 \cr 
   5_{t+1} \vspace{.6em}& \tfrac{F_0d_5}{2} & F_1d_5 & F_2d_5 & F_3d_5 & F_4d_5 & 0 }$
\end{matrix}

Thanatological age-classes will ideally terminate at the highest value permitted
by data. For the data used here, there are 111 total age classes, which
translate to 111 total remaining-years classes (0-110+). In practice $\textbf{Y}$ becomes
a 111$\times$111 matrix, with most entries non-zero. Construction may appear
tedious for this reason. However, note that the bulk of fertility entries can
be derived as the outer (tensor) product $d_a \otimes f_y$, leaving only the 
first row and first column mortality discounting followed by the addition of the
survival superdiagonal. In most statistical programming languages constructing $\textbf{Y}$ entails only
a couple more lines of code than constructing a Leslie matrix.

As with Leslie matrices, the above projection matrix may be manipulated using
standard matrix techniques. Where $\textbf{p}$ is our population vector, we
project by multiplying $\textbf{Y}$ from the left: $\textbf{p}(t + 1) =
\textbf{Y}\textbf{p}(t)$. From $\textbf{Y}$, we can extract such information as 
the intrinsic growth rate, $r$ (natural log of the largest
real eigenvalue), the stable thanatological age-structure (the real part of the
eigenvector that corresponds to the largest real eigenvalue), or the pace of
convergence to stability (the ratio of the \nth{1} to the \nth{2} eigenvalues).
\footnote{See \citet[p.86-87]{caswell2001matrix}.}

\section*{Some empirical findings}
The renewal equation and the discrete projection matrix can be put
to work with data. At this time we are only ready to report some early
results, and we do not yet have explanations some of the patterns that we
report. Of our 1834 population-years, we optimize $r$ from both
\eqref{eq:thanoren} and \eqref{eq:lotka}\footnote{It is possible to optimize $r$ using a variety of approximations, or by using a generic optimizer. We have used the method proposed by \citet{coale1957new} for Lotka's $r$ and a modified version of the same for
the thanatological $r$. Details and / or implementation code available on
request.}. In this sample, both versions of $r$ are only plausibly equal in a
single instance. Usually thantological $r$ is greater than Lotka's $r$ (1373
cases). When Lotka's $r$ is positive (693 cases), thanatological $r$ is greater
just over of 50\% of the time (356), but when Lotka's $r$ is negative,
thanatological $r$ is the greater of the two around 90\% of the time (1017
cases). These two approximations of $r$ are of opposite sign 138 times. We
provide a comparison of the $r$ distributions in Figure~\ref{fig:rDist}. Mean
locations for each distribution are indicated with vertical dashed lines; thano.
-0.0010; chrono. -0.0027. The distribution of thanatological $r$ is more compact
than, with the ratio of variances (thano./chrono.) of about 0.75. Usually the
two theoretical values of $r$ move in the same direction, but thanatological $r$
is over time the less erratic of the two, and it usually paints a less dire picture when both are negative.

\begin{figure}[h!]
	\caption{Distribution of $r$, chronological (Lotka) and thanatological*.}
	\begin{center}
		\label{fig:rDist}
		\includegraphics[scale=.7]{Figures/rDist.pdf}
	\end{center}
	\begin{tiny}
     * Data from HMD and HFD. Countries and years listed in
     Figure~\ref{fig:Fxcompare}.
	\end{tiny}
\end{figure}

From the projection matrix, the ratio of the largest to the second largest
eigenvalue, the damping ratio, is an indicator of the \texit{springiness} of the
stable population structure, $c_y$ or $c_a$, with respect to disturbances from the stable state
as determined by vital rate distributions. A higher damping ratio means that the
population structure oscillates back to its stable state faster, i.e., that
oscillations decrease in size more rapidly. The thanatological damping ratio was
greater than the Leslie damping ratio for all 1834 population-years included in
our empirical analysis. Leslie damping ratios ranged from 1.01258 to 1.0518,
while thanatological damping ratios ranged from 1.0455 to 1.0868.

* this is a work in progress, and by the time the EPC comes around I'll
have shifted the focus away on comparing stable results estimated from data
(the last couple figures), and more toward the reversibility of chronological
and thanatological stable age structures, as well as a comparative
eigen-analysis of the Leslie and thanatological projection matrices.

\vspace{2em}

\appendix
\section{Unique solution for thanatological $r$}
\label{app:A}
The solution for the intrinsic growth rate, $r$, is unique for the case of the
thanatological renewal model, \eqref{eq:thanoren}, and can be proven so in
essentially the same fashion as those in existence for the Lotka-Euler model,
\eqref{eq:lotka}. This little proof follows that given in
\textit{pressat1972demographic}. Define a convenience function, $I(r)$, for the
integrand of \eqref{eq:thanoren} for a given $r$ and fixed $F(y)$ and $d(a)$:

\begin{equation}
I(r) = \int_{y=0}^\infty \int_{a=0}^\infty F(y) d(a+y)e^{-ra}\dd a \dd y
\end{equation}
Since the death distribution function, $d(a)$, and fertility
function, $F(y)$, are continuous and non-negative, $\lim_{r \to +\infty} I(r)
= 0$ and $\lim_{r \to -\infty} I(r)= \infty$. If $r_2 > r_1$, then $I(r_1) >
I(r_2)$. $I()$ is therefore a continuous and monotonically decreasing function
of $r$ with boundaries that include the value 1 of \eqref{eq:thanoren}, and
 necessarily only obtain this value once.

\nocite{HMD,HFD}
% --------------------------------------------------
% bibliography
\bibliographystyle{plainnat}
  \bibliography{References}  

\end{document}
