


\documentclass{article}

\usepackage{amsmath}
\usepackage{natbib}
\bibpunct{(}{)}{,}{a}{}{;} 
\newcommand{\dd}{\; \mathrm{d}}
\usepackage{url}


\begin{document}
\title{Reply to Reviewers of submission 3043: ``Renewal and stability in populations structured by remaining years of
life''}
\author{Author Redacted}
\maketitle

I thank reviewers A and B for taking the time to engage with and comment on this
submission, and the assistant editor for his suggestions and guidance, and to
all involved for a swift first round of review. I've made an attempt to increase
appeal by reducing unneeded sentences and statements, and by changing
language, for instance swapping out thanatological age for years left in most
cases. Reviewers comments are then treated individually.

However, I first wish to politely state that I think DR is a good home for
purely theoretical or mathematical investigations because (in general and in no
order) a) DR has no space limitations, b) readers of DR come in almost
exclusively via content-based internet searches rather than periodical
readership (ergo there is no need to profile readership), c) the alternatives
(e.g., Theoretical Population Biology and Mathematical Population Studies) are not open access and so I have ethical objections to
them, and d) DR already sponsors a formal relationships series (ergo formal demography is demonstrably in-universe). I
should like to think that DR covers the full breadth of researchers that
demographers do. This article is standard fare for formal demography, and formal
demography is the foundation of the discipline. Browsing through the titles of
recently publishes articles in DR, I'm lead to believe that appealing to a broad
audience is not itself a criterion for publication, and I do not believe that it
should be either.

That said, I've followed advice to make the article more palatable and
approachable, and these offset somewhat the space saved by following through on
reviewer A's wish for the article to be more compact.

Comments in order:

Editor:



\end{document}







